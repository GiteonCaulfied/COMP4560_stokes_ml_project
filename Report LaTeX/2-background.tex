\chapter{Background}\label{chap:background}


\section{Neural networks: Fully connected, convolutional and long short-term memory}

In this section, we introduce some basics about neural networks (NNs) according to a more detailed explanation provided by Bishop. \citep{10.1117_1.2819119} A simple NN usually consists of one input layer, one output layer and one or more hidden layers in between. Multiple nodes inside each layers are called neurons. 

For a fully connected neural network (FNN) using linear layers for connection (the term “fully connected” means that the neurons in each layer have a connection to all neurons in the previous layer and also the following layer), each neuron receives all inputs from the previous layer and the outputs are as follows:

\begin{equation}
z = g(xA^{T} + b)
\end{equation}

In this case, $z$ represents a vector contains the result values of all the nodes in the current layer and $x$ represents a vector contains the values of all the nodes in the previous layer. $g()$ represents an activation function allows the NN to model non-linearity. $A$ is the weight matrix (with a size of $num_\text{output neurons} \times num_\text{input neurons}$) and $b$ (with a size of $1 \times num_\text{output neurons}$) is the bias vector from the linear layer (It is called a linear layer since it applies an affine map to the incoming data). Therefore, the network with a total of $n$ hidden layers, in general, can be represented as follows:

\begin{equation}
\mathcal{N}(\theta) = \Theta_\text{n} \circ g \circ \Theta_\text{n-1} ... \circ g \circ \Theta_1
\end{equation}

Here, the neural network $\mathcal{N}$ is represented as the composition of affine maps $\Theta$ and activation function $g$, where $\theta$ represents all the learnable parameters ($A$s and $b$s).

Both $A$ and $b$ are able to be optimized through a technique called error backpropagation, where we first define a loss function calculate the loss value (error) between the predicted output from the output layer and the actual output from the data set. The error is then propagated backwards through all the hidden layers using chain rule to perform differentiation. Eventually, these derivatives of errors with respect to the weights are used to update the learnable parameters ($A$ and $b$) in each hidden layer. This complete process is called gradient descent. 

By looping over the process of feeding the input data into NN to perform prediction, calculate the loss between the prediction and the actual data, use the loss value to optimize the learnable parameters in each layers, the NN model is able to be adjusted to a state that best fits the underlying pattern of the provided data set given its current structure.

In this study, we use a ML library called PyTorch to define the NN architecture, specifying the loss function and optimizer that minimize the loss function (Adam optimizer, in this case, is used throughout the study due to its broader adoption for deep learning) and systematically train and test the performance of the networks.

Apart from FNNs, Convolutional neural networks (CNNs) are also one of the most commonly used NNs. They can handle matrix or image input better than the traditional FNN with linear layers. Instead of linear layers, they use convolutional layers that contains a specified number of trainable filters, each of which is used to enhance or identify a particular feature in the input. Filters are convolved with the input image or the output of the previous layer, and the results are summed together along with a trainable bias and passed through an activation function to produce a feature map. Because convolution is also a linear operation, activation function is added to introduce non-linearity to the feature map like FNN.

In this study, a variation of CNN is used as a way to compress the size of the input data in the mantle convection problem, which is called Convolutional Autoencoder (ConvAE). It is constructed as two separate sturcture that trained together: an encoder using convolution operation to reduce the size of the original input field and output a latent space representation, and a decoder using deconvolution operation to transform the latent space representation back to the original size field. By feeding the input field into a ConvAE, the dimension of the original high-resolution fields can be decreased with the main features captured (some information may be lost during the encoding process), thus making it more computationally efficient to work with before we feed it into a prediction NN.

Since the data set in the mantle convection problem is a time sequence with adaptive timestamps, long short-term memory (LSTM) is used to predict a sequence of output apart from FNN. LSTM's architecture that use a sequence of input recurrently during prediction allows it to handle time-series data more accurately than other networks since it uses a set of previous time-steps to predict the next set of time-steps, thus leading to a potential better result.


\section{Related works for solving geoid and mantle convection using Neural networks}

Neural networks has been increasingly used for studying the geodynamics nowadays, especially when it comes to solving geoid or mantle convection simulation. 

For example, Kerl provide a bold attempt at using ML as a low-cost solution to the geoid inverse problem in his thesis, where two separate solutions using different number of CNNs are compared.\citep{kerl2022geoid} He found that the single network solution where a radial viscosity profile is predicted directly from the geoid and density data allows him to obtain a smooth, long-wavelength estimate of the Earth’s radial viscosity profile. This provides us some implications on using only one NN to solve the inverse problem instead of stacking NNs together.

As for the mantle convection problem, a group of researchers \citep{10.1093_gji_ggaa234} make use of the FNN architecture to build a surrogate model that can predict the entire evolution (0–4.5 Gyr) of the 1D temperature profile of a Mars-like planet for a wide range of values of five different parameters, including reference viscosity, activation energy and activation volume of diffusion creep, enrichment factor of heat-producing elements in the crust and initial temperature of the mantle.

In another study that is particularly worth highlighting, Agarwal and other researchers extend their previous approach \citep{10.1093_gji_ggaa234} of using FNN trained using a large number of 2D simulations to predict the evolution of entire 1D laterally-averaged temperature profile in time for complex models. Instead of predicting 1D temperature field, the full 2D temperature field are predicted since it could contain more information related to the structure of the convection.\citep{10.1103_physrevfluids.6.113801} To show that NN techniques can produce reliable parameterized surrogates, they first use ConvAE to compress the size of each temperature field by a factor of 142 and then use FNN and LSTM to predict the compressed fields. They discovered that LSTM capture the flow dynamics better than FNN despite the fact that LSTM has a lower mean relative accuracy. Their study provides us some essential insights in solving the mantle convection problem by using NNs as a low-cost solution, including using ConvAE to compress the data and compare the prediction result of two different architectures (FNN and LSTM).

