\chapter{Introduction}

Introduction (actually more like background) to Geoid problem and Mantle Convection (TODO...)

The rest of the paper is organized as follows. In Chapter 2, we briefly introduce the basic idea of neural networks (NN) and some specific architecture we use in this study. Some related works that use NNs to model the Geoid or mantle convection problem are also discussed in this chapter. In Chapter 3, we approach the Geoid problem using a simple fully connected neural network (FNN) and present the results. Then in Chapter 4, we present how we model the mantle convection problem by first compressing the temperature fields using a convolutional autoencoder (ConvAE) and then predict the compressed temperature fields using two different Machine Learning (ML) architecture - FNN and LSTM. In the same chapter, we also discussed the difference between these two architectures by visualising the result as GIF files and applying Principle Component Analysis (PCA). In the final chapter, we conclude this paper by offering some potential follow-ups about modelling mantle convection simulations using neural networks.



