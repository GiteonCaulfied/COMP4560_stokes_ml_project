\chapter{Introduction}

Geoid, in the field of geodynamics, represents the shape of a gravity equipotential across the whole of Earth's surface. Detailed maps of geoid elevation with contours spaced as close as 1m can show the effect of tectonic features such as deep sea trenches, outer arc rises, and oceanic ridges and troughs on the height of the geoid, which could provide powerful constraints that must be satisfied by the next generation of geodynamic models along with other data such as high quality determinations of surface deformation and new high resolution seismic data. \citep{10.1038_299104a0}

Geoid is generally related to both the laterally varying density of the Earth and the heterogeneity in the Earth's viscosity. However, a recent research shows that modelling that treats the geoid as being purely based on density variation near the surface can poorly explain the accurately observed geoid measurements with a satellite. \cite{10.1098_rsta.1989.0038}

(TODO: Mantle Convection Background)

Therefore, in this research, we aim to use Neural Network as a way to forward model the geoid problem,
assuming that viscosity varying in the radial direction with depth is the most important factor.

(TODO: Mantle Convection Research Aims)

The rest of the paper is organized as follows. In Chapter 2, we briefly introduce the basic idea of neural networks (NN) and some specific architecture we use in this study. Some related works that use NNs to model the Geoid or mantle convection problem are also discussed in this chapter. In Chapter 3, we approach the Geoid problem using a simple fully connected neural network (FNN) and present the results. Then in Chapter 4, we present how we model the mantle convection problem by first compressing the temperature fields using a convolutional autoencoder (ConvAE) and then predict temporal evolution of the compressed temperature fields using two different Machine Learning (ML) architecture - FNN and LSTM. In the same chapter, we also discussed the difference between these two architectures by visualising the result as GIF files and applying Proper Orthogonal Decomposition (POD). Further testing on FNN given an interpolated data set are conducted to determine experimentally for how many time steps we can use the trained FNN during
a set of S consecutive time step without loosing track of the transient dynamics. In the final chapter, we conclude this paper by offering some potential follow-ups about modelling mantle convection simulations using neural networks.



