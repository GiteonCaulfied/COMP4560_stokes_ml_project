\chapter{Introduction}

Earth’s mantle, between the crust and the liquid outer core, undergoes a process known as mantle convection. This is characterised by the slow churning motion driven by heat originating from the core. It entails the upward movement of warm, buoyant material and the downward migration of cooler, denser matter. This phenomenon is the main driving force behind tectonic plate movements and sculpting the Earth’s surface. Intriguingly, despite the mantle being solid rock, it exhibits viscous fluid-like behaviour over vast geological timescales spanning millions of years.

A paramount challenge in geosciences is reconstructing the thermochemical evolution of Earth’s mantle and its various surface manifestations. This necessitates the development of a digital twin—a digital depiction of Earth’s mantle across spatial and temporal dimensions, harmonising with observational data pertaining to the mantle’s structure, dynamics, and evolution. Notably, while the continuum equations governing mantle convection are understood, it remains an initial condition problem, implying that insights into previous mantle states are imperative for reconstructing Earth’s evolution.

To address this issue, the concept of the inverse problem is introduced. Here, the aim is to determine a model through gradient-optimisation techniques, which aligns most coherently with data derived from partial, imperfect, or noisy measurements. The predominant strategy for deciphering this inverse problem is the adjoint method. Nevertheless, this approach is computationally intensive, necessitating the computation of both a time-dependent nonlinear forward problem and the corresponding adjoint system for each optimisation iteration.

The primary motivation behind this research is to curtail computational requirements. This is achieved by exploring the potential of deep learning methods as surrogate models. These models offer the means to approximate the forward and adjoint problems with significantly reduced computational cost.

As a first attempt in this direction, we have addressed the forward problem associated with two well-established geodynamics problems: 

\begin{enumerate}
    \item The semi-analytical solutions to the system integrated with Stokes and Poisson gravitational equations—referred to as the ‘geoid’ problem. These solutions simplify the Stokes system by presuming radial symmetry in the mantle’s viscous properties, resulting in an instantaneous system.

    \item The 2D Mantle convection problems. Here, the thermal evolution of a Stokes system is modelled by resolving the partial differential equations derived from the conservation equations of mass, momentum, and energy, in their simplest form of the Boussinesq approximation. 
\end{enumerate}

The rest of the paper is organized as follows. In Chapter 2, we briefly introduce some basic concepts of Neural Networks (NN) and some specific architecture we use in this study. Some related works that use NNs to model the geoid or mantle convection problem are also discussed in this chapter. In Chapter 3, we approach the geoid problem using a simple Fully Connected Neural Network (FNN) and present the results. Then in Chapter 4, we present how we model the mantle convection problem by first compressing the temperature fields using a Convolutional AutoEncoder (ConvAE) and then predict temporal evolution of the compressed temperature fields using two different Machine Learning (ML) architecture - FNN and Long Short-Term Memory (LSTM). In the same chapter, we also discussed the difference between these two architectures by visualising the result as animations save in GIF format and applying Proper Orthogonal Decomposition (POD). Further testing on FNN given an interpolated dataset are conducted to determine experimentally for how many time steps we can use the trained FNN during
a set of S consecutive time step without loosing track of the transient dynamics. In the final chapter, we conclude this paper by offering some potential follow-ups about modelling geoid problems and mantle convection simulations using neural networks.



