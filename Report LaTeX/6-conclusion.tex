\chapter{Concluding Remarks}\label{chap:conclusion}

\section{Conclusion}

In this research, we use Neural Networks (NN) to solve the forward modelling problem of geoid as a set of spherical harmonic coefficients given a 1D spherically symmetric viscosity model. The data set used to train the algorithms consists of 1000 pairs of the above models and has a reduced prior
such that the perturbations to the output are smaller to simplify the problem. The data set is feed into different Feed Forward Neural Network (FNN) architectures with different set of hyperparameters systematically to find the model with the best performance. 

We found that FNN architectures with a total number 3–4 hidden layers have the best mean accuracy. This could presented as a proof showing that high-dimensional regression algorithms
like neural networks can help with the forward modelling of the geoid problem when predicting
a set of spherical harmonic coefficients given a reduced dataset with a fairly small number of
1D spherically symmetric viscosity models.

(TODO: Mantle Convection Conclusion)

\section{Future Work}

For the geoid problem, Future studies may focus on extending the use of neural networks on solving the geoid problem using viscosity models by testing with a more general data set, since the one used in this research is using a reduced prior such that the perturbations to the output are far smaller. Also, there could be more exploration on the potential use of neural networks to backward model the geoid problem.


As for the mantle convection problem, future studies may continue to explore the potential use of the trained FNN on a set of S consecutive time steps. If we can somehow replace the fine-grained truth temperature field from the simulator (size is 201x401), that is used to "correct" the predicted time series, with a coarse-grained temperature field (e.g. size is 50x100) instead, the computational complexity of modelling mantle convection problem could be further reduced. Future studies could also focus on improving the performance of LSTM or using less truth temperature fields to predict more temperature fields, rather than sticking to the current ratio of 50:50 for input and output due to the technical limitation of PyTorch.