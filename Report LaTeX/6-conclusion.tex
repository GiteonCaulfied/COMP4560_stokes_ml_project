\chapter{Concluding Remarks}\label{chap:conclusion}

\section{Conclusion}
(TODO...)

\section{Future Work}
For the mantle convection problem, future studies may continue to explore the potential use of the trained FNN on a set of S consecutive time steps. If we can somehow replace the fine-grained truth temperature field from the simulator (size is 201x401), that is used to "correct" the predicted time series, with a coarse-grained temperature field (e.g. size is 50x100) instead, the computational complexity of modelling mantle convection problem could be further reduced. Future studies could also focus on improving the performance of LSTM or using less truth temperature fields to predict more temperature fields, rather than sticking to the current ratio of 50:50 for input and output due to the technical limitation of PyTorch.